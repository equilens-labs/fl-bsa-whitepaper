
\section{Results}
\subsection{Fairness}
\begin{table}[t]
\centering
\begin{tabular}{lllSSSS}
\toprule
attribute & protected & reference & {AIR} & {LCI} & {UCI} & {p}\\
\midrule
gender & female & male & \num{0.771} & \num{0.741} & \num{0.802} & \num{0.000000}\\
\bottomrule
\end{tabular}

\caption{Adverse Impact Ratio by protected attribute on the synthetic audit test split. AIR is the minimum ratio of group-wise selection rates as defined in \cref{eq:air}, with 95\% BCa bootstrap confidence intervals, reported as lower and upper confidence limits (LCI/UCI). AIR values below the 0.80 threshold indicate a potential disparate impact under the four-fifths rule.}
\label{tab:air_summary}
\end{table}

\IfFileExists{figures/air_summary.pdf}{%
\begin{figure}[t]
\centering
\includegraphics[width=0.6\textwidth]{figures/air_summary.pdf}
\caption{Adverse impact ratio per protected attribute with 95\% BCa confidence intervals. The dashed line shows the 0.80 threshold.}
\label{fig:air_summary}
\end{figure}
}{}

\IfFileExists{figures/selection_rates.pdf}{%
\begin{figure}[t]
\centering
\includegraphics[width=0.85\textwidth]{figures/selection_rates.pdf}
\caption{Selection rates by protected attribute (gender and race) on the synthetic audit test split, with 95\% Wilson confidence intervals. These rates underlie the AIR calculations in \Cref{tab:air_summary}.}
\label{fig:selection_rates}
\end{figure}
}{}

\FloatBarrier

\subsection{Equalised Odds}
\begin{table}[t]
\centering
\begin{tabular}{llSSS}
\toprule
attribute & metric & {value} & {LCI} & {UCI}\\
\midrule
gender & tpr\_gap & \num{0.000} & \num{0.000} & \num{0.000}\\
race & tpr\_gap & \num{0.000} & \num{0.000} & \num{0.000}\\
gender & fpr\_gap & \num{0.000} & \num{0.000} & \num{0.000}\\
race & fpr\_gap & \num{0.000} & \num{0.000} & \num{0.000}\\
\bottomrule
\end{tabular}

\caption{Equalised Odds gaps (TPR and FPR differences) by protected attribute on the synthetic audit test split, with 95\% BCa bootstrap confidence intervals, reported as lower and upper confidence limits (LCI/UCI). Gaps are computed as in \cref{eq:eo-gap}, that is, as the maximum minus the minimum group-wise TPR/FPR across categories of each attribute.}
\label{tab:eo_summary}
\end{table}

\ifnum\HasDegenerateCIs>0
\noindent\emph{Note.} Some groups in this synthetic scenario have rates at 0\% or 100\%. With aggregate bootstrap this can yield zero-width confidence intervals, even though real-world uncertainty would be larger. FL-BSA compliance reports use row-level BCa bootstrap to mitigate this effect.
\fi

\FloatBarrier

\subsection{Calibration}
\begin{table}[t]
\centering
\begin{tabular}{llllSS}
\toprule
run & model & split & {ECE} & {LCI} & {UCI}\\
\midrule
\multicolumn{6}{c}{\emph{No ECE rows found in metrics}}\\
\bottomrule
\end{tabular}

\caption{ECE summary for this run. ECE is defined in \cref{eq:ece}. In this synthetic audit scenario, ECE is not evaluated because no probability scores are available in the intake bundle; calibration compliance is therefore unknown.}
\label{tab:ece_summary}
\end{table}

\ifnum\EceEvaluated>0\relax
% Calibration evaluated; no additional note.
\else
\noindent\emph{Note.} Calibration (ECE) was not evaluated in this scenario; calibration compliance is therefore unknown in this whitepaper.
\fi

\FloatBarrier

\subsection{Provenance}
The run manifest records the dataset hash, code commit, container image digests, random seeds, and timestamps; key fields are listed in Appendix~E.
