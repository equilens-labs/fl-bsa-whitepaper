
\section{Problem and Estimands}
We use the following notation:
\begin{itemize}
\item $A$ denotes a protected attribute (for example, gender or race), with group values $g\in\mathcal{A}$;
\item $Y\in\{0,1\}$ denotes the true label (e.g.\ good \emph{vs.} bad credit outcome);
\item $S\in[0,1]$ denotes a score interpreted as an estimated probability; and
\item $\hat{Y}=\mathbf{1}[S\ge t]$ denotes the binary decision obtained by thresholding the score at level $t$.
\end{itemize}

\subsection{Adverse Impact Ratio}
\begin{equation}
\mathrm{AIR} \;=\; \min_{g\in\mathcal{A}} \frac{\Pr(\hat{Y}=1\mid A=g)}{\Pr(\hat{Y}=1\mid A=g^\star)}\,,\quad
g^\star=\arg\max_g \Pr(\hat{Y}=1\mid A=g).
\label{eq:air}
\end{equation}
This corresponds to the regulatory four-fifths rule for disparate impact \citep{eeoc_uniform_guidelines}.

\subsection{Equalised Odds}
For each group $g\in\mathcal{A}$, the true positive rate (TPR) and false positive rate (FPR) are defined as
\begin{align}
\mathrm{TPR}_g &= \Pr(\hat{Y}=1\mid Y=1,A=g),\\
\mathrm{FPR}_g &= \Pr(\hat{Y}=1\mid Y=0,A=g).
\end{align}
Gaps are summarised as
\begin{equation}
\mathrm{TPR\_gap}=\max_g \mathrm{TPR}_g-\min_g \mathrm{TPR}_g
\label{eq:eo-gap}
\end{equation}
and analogously for FPR, following the equalised-odds criterion of \citet{hardt2016equality}.

\subsection{Calibration}
For bins $B_k$ with counts $n_k$, mean score $\bar s_k$, and empirical positive rate $\hat p_k$, the expected calibration error (ECE) is
\begin{equation}
\mathrm{ECE} \;=\; \sum_k \frac{n_k}{\sum_j n_j} \,\lvert \hat p_k - \bar s_k\rvert.
\label{eq:ece}
\end{equation}
We use ECE as a scalar summary of calibration quality, in line with modern empirical calibration work \citep{guo2017calibration}.
